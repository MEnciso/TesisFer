
\chapter*{Agradecimientos}\label{agradecimientos_chapter}  
%\addcontentsline{toc}{chapter}{\protect{Agradecimientos}}
\thispagestyle{empty}
\markboth{Agradecimientos}{Agradecimientos}
%Agradecimientos
%
%dadsadsad
%
%as
%dasdas
%d
%sa
%dsad
%
%
%
%sd
%as
%dsa
%d
%
%
%
%adsasda

%



\noindent
A menudo he escuchado que esta es la parte que m�s dif�cil resulta redactar; coincido con ello, pero tambi�n tengo claro que es la que m�s he disfrutado escribiendo. 

\vspace{0.3cm}

Ahora que ya se termina esta etapa de mi vida con la consecuci�n de esta tesis doctoral, es momento de hacer balance y de agradecer a las personas que han intervenido en este gran logro; porque es tambi�n gracias a ellos.

En primer lugar, quiero agradecer a mis queridos directores, Dr. Manuel Enciso y Dr. Carlos Rossi, la direcci�n de la tesis y todo lo que me hab�is ense�ado, pero en especial, vuestro trato, con el que siempre me hab�is hecho sentir uno de vosotros, y donde la acogida siempre ha sido la mejor; por todo ello, gracias.

Manolo, me has dirigido: el proyecto fin de carrera de la ingenier�a t�cnica de sistemas, el proyecto fin de carrera de la ingenier�a superior, el trabajo fin de m�ster, y ahora, la tesis doctoral. Quiero decirte que me siento orgulloso de ello, y que si tuviera que volver a recorrer todo ese camino, volver�a a pedirte que estuvieras de nuevo a mi lado; por todo ello, gracias.

Carlos, lo primero es decirte que ojal� nos hubi�ramos cruzado antes. Adem�s de por la direcci�n de la tesis, tambi�n quiero darte las gracias por haber buscado siempre posibilidad de financiaci�n para que pudiera afrontar este periodo de investigaci�n con un respaldo; por todo ello, gracias.

Quiero hacer menci�n especial al Dr. Sergio G�lvez. Sergio, ha sido una gran suerte tenerte cerca para dudas y consejos, y en realidad, por el simple hecho de conversar contigo. Siempre me has apoyado y valorado mucho, y eso significa mucho para m�; por todo ello, gracias.

Gracias a mis coautores: Dr. Pablo Cordero, Dr. �ngel Mora, Dr. Antonio Guevara, ha sido un placer publicar a vuestro lado; a la Dra. Llanos Mora como tutora de tesis; al Dr. Antonio Vallecillo por buscar un contacto en el momento preciso; al Dr. Dar�o Guerrero y al Dr. Rafael Larrosa por el apoyo en el Centro de Supercomputaci�n.

Ahora, quiero darte las gracias a ti, mi querida Aurora, por ser mi apoyo y �nimo diario e incondicional, por quererme siempre cada d�a y soportar mi insaciable inquietud, y por haber vivido juntos todos estos �xitos que la vida nos est� deparando, que seguro, a tu lado, ser�n muchos m�s.

Lo primero que quiero decirle a mi padre y a mi madre es: lo he conseguido. Gracias por darme el mejor entorno familiar posible, por vuestro interminable esfuerzo y por inculcarme la vital importancia de la educaci�n y el saber. Ahora es momento de recoger los frutos y disfrutar de lo que hemos conseguido todos juntos. 

Gracias tambi�n a ti, hermano, por ser la persona que m�s me ha ense�ado en la vida y de quien sigo aprendiendo a cada momento juntos. Gracias por estar siempre disponible y por encauzarme y guiarme por el infinito camino de la Ciencia.

Quiero hacer un gui�o a mis profesores de la ni�ez, D. Justo, D. Juli�n, D. Luis y D. Andr�s, por construir a tan temprana edad, unos s�lidos cimientos sobre los que crecer.

Seguramente olvide alguien, no obstante, gracias a toda persona que hasta el d�a de hoy me haya ense�ado cualquier cosa, pues para m�, lo m�s importante es el Conocimiento.


%\pagestyle{headings}
 

%%%%%%%%%%%%%%%%%%%%%%%%%%%%%%%%%%%%%%%%%%%%%%%%%%%%
